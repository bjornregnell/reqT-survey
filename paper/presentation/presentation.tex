\documentclass{slides}

\setbeamertemplate{footline}[frame number]
\title[What is essential?]{What is essential?}
\subtitle{\footnotesize -- A pilot survey on views about the requirements metamodel of reqT.org}
\author{Björn Regnell}
\institute{Lund University}
\date{March 14, 2016\\ \href{http://refsq.org/2016}{refsq.org/2016}\\ \href{https://github.com/bjornregnell/reqT-survey}{github.com/bjornregnell/reqT-survey}}

\begin{document}

\frame{\titlepage}
\frame{\tableofcontents}

\section{Objective}
\subsection{Research Question}
\begin{Slide}{Research question}
In the context of software requirements engineering education:
\begin{itemize}
\item How to choose a set of \\ \Emph{essential requirements engineering concepts} \\ that allows for sufficient expressiveness, \\ without overloading the metamodel with esoteric concepts just for the sake of completeness?
\end{itemize}
\end{Slide}
\subsection{Approach}
\begin{Slide}{Approach}
Make a survey among RE scholars:
\begin{itemize}
\item How to quantify ''essentiality''?
\pause
\item One possible quantification: \\ The more scholars that agree on a definition of a concept \\ and \\ the more scholars that use the concept, \\ the more essential is the concept.
\end{itemize}
\end{Slide}


\end{document}

