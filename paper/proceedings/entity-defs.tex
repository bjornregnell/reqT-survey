\texttt{Actor}&A human or machine that communicates with a system.\\
\texttt{App}&A computer program, or group of programs designed for end users, normally with a graphical user interface. Short for application.\\
\texttt{Barrier}&Something that makes it difficult to achieve a goal or a higher quality level.\\
\texttt{Breakpoint}&A point of change. An important aspect of a (non-linear) relation between quality and benefit.\\
\texttt{Class}&An extensible template for creating objects. A set of objects with certain attributes in common. A category.\\
\texttt{Component}&A composable part of a system. A reusable, interchangeable system unit or functionality.\\
\texttt{Configuration}&A specific combination of variants.\\
\texttt{Data}&Information stored in a system.\\
\texttt{Design}&A specific realization or high-level implementation description (of a system part).\\
\texttt{Domain}&The application area of a product with its surrounding entities.\\
\texttt{Epic}&A large user story or a collection of stories.\\
\texttt{Event}&Something that can happen in the domain and/or in the system.\\
\texttt{Feature}&A releasable characteristic of a product. A (high-level, coherent) bundle of requirements.\\
\texttt{Function}&A description of how input data is mapped to output data. A capability of a system to do something specific.\\
\texttt{Goal}&An intention of a stakeholder or desired system property.\\
\texttt{Idea}&A concept or thought (potentially interesting).\\
\texttt{Interface}&A defined way to interact with a system.\\
\texttt{Issue}&Something needed to be fixed.\\
\texttt{Item}& An article in a collection, enumeration, or series.\\
\texttt{Label}&A descriptive name used to identify something.\\
\texttt{Member}&An entity that is part of another entity, eg. a field in a in a class.\\
\texttt{Meta}&A prefix used on a concept to mean beyond or about its own concept, e.g. metadata is data about data.\\
\texttt{MockUp}&A prototype with limited functionality used to demonstrate a design idea.\\
\texttt{Module}&A collection of coherent functions and interfaces.\\
\texttt{Product}&Something offered to a market.\\
\texttt{Quality}&A distinguishing characteristic or degree of goodness.\\
\texttt{Relationship}&A specific way that entities are connected.\\
\texttt{Release}&A specific version of a system offered at a specific time to end users.\\
\texttt{Req}&Something needed or wanted. An abstract term denoting any type of information relevant to the (specification of) intentions behind system development. Short for requirement.\\
\texttt{Resource}&A capability of, or support for development.\\
\texttt{Risk}&Something negative that may happen.\\
\texttt{Scenario}&A (vivid) description of a (possible future) system usage.\\
\texttt{Screen}&A design of (a part of) a user interface.\\
\texttt{Section}&A part of a (requirements) document.\\
\texttt{Service}&Actions performed by systems and/or humans to provide results to stakeholders.\\
\texttt{Stakeholder}&Someone with a stake in the system development or usage.\\
\texttt{State}&A mode or condition of something in the domain and/or in the system. A configuration of data.\\
\texttt{Story}&A short description of what a user does or needs. Short for user story.\\
\texttt{System}&A set of interacting software and/or hardware components.\\
\texttt{Target}&A desired quality level or goal .\\
\texttt{Task}&A piece of work (that users do, maybe supported by a system).\\
\texttt{Term}&A word or group of words having a particular meaning.\\
\texttt{Test}&A procedure to check if requirements are met.\\
\texttt{Ticket}&(Development) work awaiting to be completed.\\
\texttt{UseCase}&A list of steps defining interactions between actors and a system to achieve a goal.\\
\texttt{User}&A human interacting with a system.\\
\texttt{Variant}&An object or system property that can be chosen from a set of options.\\
\texttt{VariationPoint}&An opportunity of choice among variants.\\
\texttt{WorkPackage}&A collection of (development) work tasks.\\
